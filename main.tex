\documentclass{article}
\usepackage{graphicx}
\usepackage[utf8x]{inputenc}
\usepackage[english,russian]{babel}
\usepackage[T2A]{fontenc}
\usepackage[12pt]{extsizes}
\usepackage{amsmath}
\usepackage{amssymb}
\usepackage{amsthm}
\usepackage{comment}
\usepackage{setspace}

\title{Теоретический минимум для экзамена по Дополнительным Главам Уравнений в Частных Производных}
\author{Х. С. Горы}
\date{Январь 2025 или типа того}

\begin{document}

\maketitle
\centering
\newtheorem{definition}{Опр.}[subsection]
\newtheorem{theorem}{Теор.}[subsection]

\onehalfspacing

\raggedright
\setcounter{section}{-1}
\section{Оправдание}

Я вызвался писать этот документ будучи относительно пьяным. В трезвом уме я бы ни за что не согласился тратить на это время, потому что мне ультралень. Однако here we are.
Пока что формат документа планируется следующий: просто переписать все определения и формулировки из каждой записанной лекции А.Б.Костина. \\
\vspace{14}
Поплакать об оформлении, попросить письменные конспекты, попросить в долг 500 рублей: t.me/diracseascrolls
\vspace{14}
% 13.01.25 1:30. Я начинаю писать эту ботву в надежде что допишу хотя бы одну логическую тему. \\
% 13.01.25 13:30. :( \\
% 13.01.25 18:30. Написал 1 главу. \\
% 13.01.25 21:00. :( \\

\section{Глава: Безымянная}

$\diamond$ Лекция 1

\subsection{Общие понятия и обозначения.}\\

$\Omega$ -- область в $\mathbb{R}^{n}$,\, $x = (x_1, \dots, x_n) \in \Omega$ \\
\vspace{14}
$\bullet$ $\alpha = (\alpha_1, \dots, \alpha_n) \subset \mathbb{Z}_+$ -- набор целых неотрицательных чисел -- называется мультииндексом. $|\alpha| = \alpha_1 + \dots + \alpha_n$ \\
$x^{\alpha} = x_1^{\alpha_1} \cdot x_2^{\alpha_2} \cdot \ldots \cdot x_n^{\alpha_n}$ \\
$\alpha!=\alpha_1 \cdot \alpha_2 \cdot \dots \alpha_n$ \\
\vspace{14}
$\bullet$ $u: \Omega \to \mathbb{R}$ ( $\mathbb{C}$ ) \\
$\mathbb{D}_{x}^{\alpha} u = \frac{\partial^{|\alpha|} u}{\partial x_1^{\alpha_1} \partial x_2^{\alpha_2} \dots \partial x_n^{\alpha_n}} = \mathbb{D}_{x_1}^{\alpha_1} \cdot \mathbb{D}_{x_2}^{\alpha_2} \cdot \dots \cdot \mathbb{D}_{x_n}^{\alpha_n} u$ \\
Уравнение в частных производных:
$$
F(x,u, u_{x_1}, \dots, u_{x_n}, \dots, \mathbb{D}_{x}^{\alpha} u, \dots) = 0
$$ 
имеет порядок m, если $|\alpha| =m$, и не входят производные более высоких порядков. \\
Линейное уравнение:
$$
\sum_{\alpha:|\alpha| \le m} a_{\alpha}(x)\mathbb{D}_{x}^{\alpha} u = 0 \\
$$ \\
$\checkmark$ Квазилинейное уравнение - линейное относительно старших производных.
\vspace{14}
\begin{definition}[Пространство $C^k(\Omega)$]
$\Omega \subset \mathbb{R}^n , u : \Omega \to \mathbb{R}, k \in \mathbb{Z}_{+}$ \\
$u \in C^{k}(\Omega)$, если $ u \in C(\Omega)$, и во всех внутренних точках $\Omega$ существуют все частные производные до порядка k включительно, которые по непрерывности продолжнаются на всё $\Omega.$ \\
\end{definition}

При выводе уравнений теплопроводности и равновесия мембраны получаются интегральные тождества. Решение такого тождества называется обобщённым решением. \\

\subsection{Распространение тепла в среде.} 

$G \subset \mathbb{R}^3$, $x \in G$, $u(x,t)$ - температура в точке. \\
$\rho$ - плотность, $c$ - удельная теплоёмкость, $k(x)$ - коэффициент (внутренней) теплопроводности. \\
% $n$ - внешняя нормаль, $|\overrightarrow{n}| = 1$ \\
\vspace{14}
% $Q_1 = \int_{t}^{t+\Delta t} dt \int_{\partial \Omega} k \frac{\partial u}{\partial n} ds_x$
% $Q_2 = \int_{t}^{t+\Delta t} dt \int_{\Omega} F(x,t) dx$
% $Q_3 = \int_{\Omega} c\rho(u(x, t+\Delta t) - u(x, t))dx $

Уравнение теплопроводности в интегральной формулировке:
$$
\int_{t}^{t+\Delta t} dt \int_{\Omega} (\operatorname{div}(k \nabla u) + F - c \rho \frac{\partial u}{\partial t}) dt = 0,\; \forall \Omega \ \forall \Delta t > 0
$$
В дифференциальной формулировке:
$$
c\rho \frac{\partial u}{\partial t} - \operatorname{div}(k \nabla u) = F,\; \forall t > 0 \, \forall x \in G
$$
Начальные и граничные условия:
$u(x, 0) = u_0(x), x \in G$ - \text{нач. усл.} \\
$\bullet$ Граничные условия 1 рода (температура): $u(x,t) = u_1(x,1), x \in \partial G, t \ge 0 $\\
$\bullet$ Граничные условия 2 рода (поток): $k \frac{\partial u}{\partial n} = u_2(x,t), x \in \partial G, t \ge 0 $\\
$\bullet$ Граничные условия 3 рода (теплообмен): $k \frac{\partial u}{\partial n} + h(u - u_e) = 0$ \\
\vspace{14}
$\diamond$ Лекция 2

\subsection{Равновесие мембраны.}

Упругая тонкая плёнка. Проектируется на область $\Omega$ плоскости $(x_1,x_2)$.
\vspace{14}
Потенциальная энергия для свободной границы: \\
$U = C + \frac{1}{2}\int_{\Omega} (k|\nabla u|^2 + a u^2 + 2 fu) dx +\frac{1}{2}\int_{\partial \Omega} (a_1u^2-2f_1u) ds_x $ (1) \\
(здесь $C=(U_0 - A_1^0-A_2^0-A_3^0)$) \\
Условие закрепления границы: $u(x) = \varphi(x) \; x \in \partial \Omega$ (2) \\
Потенциальная энергия с фиксированной границей: \\
$U = C_1 + \frac{1}{2} \int_{\Omega} (k|\nabla u|^2 + au^2 - 2fu)dx$ (3) \\
\vspace{14}
Минимизируя какой-то функционал, получаем интегральное тождество (обобщённая формулировка задачи равновесия мембраны):
$$
\int_{\Omega} (k(\nabla u, \nabla v) + auv - fv)dx = 0,\; \forall v \in C^1(\Omega):v|_{\partial \Omega} = 0
$$
Для задачи со свободной границей:
$$
\int_{\Omega} (k(\nabla u, \nabla v) + auv - fv)dx + \int_{\partial \Omega} (a_1uv - f_1v)ds_x = 0,\; \forall v \in C^1(\Omega):v|_{\partial \Omega} = 0
$$
$\checkmark$ Можно показать, что эти задачи эквивалентны. \\
\vspace{14}
Уравнение равновесия мембраны в дифференциальной форме:
\begin{equation}
\begin{cases}
$-\operatorname{div}(k \nabla u) + a \cdot u = f ,\; x \in \Omega$ \\
$u(x) = \varphi(x),\; x \in \partial \Omega$ \\
\end{cases}
\end{equation}
Для задачи со свободной границей:
\begin{equation}
\begin{cases}
$-\operatorname{div}(k \nabla u) + a \cdot u = f ,\; x \in \Omega$ \\
$ k\dfrac{\partial u}{\partial n} + a_1u = f_1,\; x \in \partial \Omega$ \\
\end{cases}
\end{equation}

$\diamond$ Лекция 3

\subsection{Колебания мембраны.}
% ???
% объяснить, что такое u
Интегральное тождество, описывающее колебания мембраны: \\
$$
\int_{t_1}^{t_2} \int_{\Omega} (\rho u_t v_t - k(\nabla u, \nabla v) - auv + fv) dx dt -\int_{t_1}^{t_2} \int_{\partial \Omega} (a_1u - f_1)v ds_x dt = 0,\;\\
\forall t_1,t_2,\ \forall v \in C^1(\bar{\Omega} \times [0,+\infty)), v|_{t=t_1}=v|_{t=t_2}=0
$$ \\
Дифференциальная формулировка тождества (Уравнение колебаний мембраны):
\begin{equation}
\begin{cases}
$\rho u_{tt} - \operatorname{div}(k \nabla u) + au = f,\; x \in \Omega, t > 0$ \\
\text{Н.у.: } $k\dfrac{\partial u}{\partial n} + a_1u = f_1,\; x \in \partial \Omega, t > 0$ \\ % начальные условия
\text{Гр.у.: } $u(x,0) = u_0(x),\ u_t(x,0)=u_1(x), x \in \bar{\Omega}$ \\ % граничные условия
\end{cases}
\end{equation}

$\checkmark$ Можно доказать, что эта задача эквивалентна задаче поиска минимума некоторого функционала, как для прошлой задачи. % и что?

\section{Глава: Некоторые вопросы общей теории УрЧП.}
\subsection{Задача Коши. Теорема Ковалевской.}

В этой главе (или хотя бы в этом параграфе) $n$ пространственных переменных и время: \\
$u = u(x,t), \alpha = (\alpha_0, \alpha_1, \dots, \alpha_n) \in \mathbb{Z}_+,$ \\
$\mathbb{D}^{\alpha}u = \mathbb{D}_t^{\alpha_0} \mathbb{D}_{x_1}^{\alpha_1} \dots \mathbb{D}_{x_n}^{\alpha_n} u$ \\
\vspace{14}
Ставится задача Коши для дифференциального уравнения: \\
\begin{equation}
\begin{cases}
$\dfrac{\partial^k u}{\partial t^k} = \sum_{\substack{|\alpha| \le k \\ \alpha_0 < k}} a_{\alpha}(x,t)\mathbb{D}^{\alpha}u + f(x,t)$ \\
\text{н.у.: } $ \left.\dfrac{\partial^m u}{\partial t^m} \right|_{t=t^0}=\varphi_m(x),\ \forall m={0,1,\dots,k-1} $ \\
\end{cases}
\end{equation}
$\checkmark$ Такое уравнение называют линейным уравнением типа Ковалевской.
\begin{definition}[]
Пусть $F : \Omega \to \mathbb{C}, \Omega \subset \mathbb{R}^n, x = (x_1,\dots, x_n) \in \Omega$. Функция $F$ называется аналитической в точке $x_0 \in \Omega$, если она раскладывается в некоторой окрестности точки $x_0$ в абсолютно сходящийся степенной ряд. \\
\end{definition}
$$
F(x) =\sum_{\alpha_1=0}^{\infty} \sum_{\alpha_2=0}^{\infty} \dots \sum_{\alpha_n=0}^{\infty} A_{\alpha1 \dots \alpha_n}(x-x_1)^{\alpha_1} \cdot \ldots \cdot (x-x_n)^{\alpha_n}
$$
В терминах мультииндексов:
$$
F(x) = \sum_{\alpha} A_{\alpha}(x-x^0)^{\alpha}
$$

\begin{theorem}[Коши-Ковалевской о локальной разрешимости и единственности решения задачи Коши]
Если функции $a_{\alpha}(x,t)$,$f(x,t)$ аналитические в окрестности т. $(x^0, t^0),\ x^0 \in \Omega, \ t^0 \in (a,b)$, а все функции $\varphi_m(x)$ аналитические в окр-ти $x^0$, то в некоторой окр-ти т. $(x^0, t^0)$ существует аналитическое решение $u(x,t)$, которое единственно в классе аналитических функций.
% решение чего? вставить сюда актуальные циферные ссылки (1),(2)
\end{theorem}

$\diamond$ Лекция 4

Система типа Ковалевской (нормальная система):
$i=1,2,\dots,N$
$$
\dfrac{\partial^n u_i}{\partial t^{n_i}} = F_i(x,t,u,\dots,\dfrac{\partial^{|\tilde{\alpha}|} u_j}{\partial t^{\alpha_0} \cdot \partial x_1^{\alpha_1} \cdot \ldots \cdot \partial x_n^{\alpha_n}},\dots),
$$
где $n_i$ - наивысший порядок производной функции $u_i$, входящей в $i$ уравнение по переменной $t$. \\
$|\tilde{\alpha}|=\alpha_0 + \alpha_1 + \dots + \alpha_n, \ \alpha_0 < n_j, \ |\tilde{\alpha}| \le n_j$ \\
Условия: 1) производная наивысшего порядка по переменной $t$ должна обязательно содержаться в i-том уравнении, 2) производная должна явно выражаться через оставшиеся. (То есть, система является разрешённой относительно старших производных по переменной $t$. \\
\vspace{14}
Начальные условия: $\frac{\partial^k u_i}{\partial t^k}(x,t^0) = \varphi_{i,k}(x),\; \forall k=0,\dots,n_i-1,\ \forall i = \overline{1..N}$ \\

\begin{theorem}[О локальной разрешимости и единственности решения задачи Коши для системы]
Пусть функции $\varphi_{i,k}$ аналитические в окрестности $x^0$, функции $F_i$ - аналитические своих аргументов в окрестности $(x^0, t^0, u(x^0,t^0), \dots, \dfrac{\partial^{|\tilde{\alpha}|} u_j}{\partial t^{\alpha_0} \cdot \partial x_1^{\alpha_1} \cdot \ldots \cdot \partial x_n^{\alpha_n}}(x^0,t^0), \dots)$. Тогда задача Коши имеет аналитическое решение в окрестности $(x^0, t^0)$, которое единственно в классе аналиических вектор-функций. 
\end{theorem}

$\diamond$ Лекция 5

\subsection{Пример отсутстия аналитического решения задачи уравнения теплопроводности}
У этой задачи не существует решения, аналитического в окрестности $(0,0)$:

\begin{equation}
\begin{cases}
$u_t = u_{xx}, \; x \in \mathbb{R},\ t>0$ \\
$u(x,0) = \frac{1}{1+x^2}, \; x \in \mathbb{R}$ \\
\end{cases}
\end{equation}
Это решение:
\begin{equation}
u(x,t)=
\begin{cases}
$\dfrac{1}{2\sqrt{\pi t}} \int_{-\infty}^{\infty} e^{-\dfrac{(x-\xi)^2}{4t}} \dfrac{1}{1+\xi^2} d\xi, \; t > 0 $ \\
$\dfrac{1}{1+x^2}, \; t = 0.$ \\
\end{cases}
\end{equation}
не является аналитическим, но является единственным в классе ограниченных функций.

\subsection{Характеристики и характеристические направления.}
% Особенности постановки задачи Коши с начальными данными на характеристиках.
Уравнение $\sum_{|\alpha| \le m} a_{\alpha}(x)\mathbb{D}^{\alpha}u = f(x)$.
Поверхность $S :F(x)=0$, $F \in C^1, |\nabla F| \ne 0 \ \forall x \in S$.
Тогда существует нормаль $n=n(x)$.

\begin{definition}[Характеристическая поверхность]
Гладкая поверхность $S$, для которой выполняется равенство $\sum_{|\alpha|= m}a_{\alpha}(x)(\frac{\partial F}{\partial x_0})^{\alpha_0}(\frac{\partial F}{\partial x_1})^{\alpha_1}\dots(\frac{\partial F}{\partial x_n})^{\alpha_n} \equiv \sum_{|\alpha|=m} a_{\alpha}(x)(\nabla F)^{\alpha} = 0$ называется характеристической поверхностью для уравнения (1), или просто характеристикой. \\
\end{definition}
$\checkmark$ Коэффициент перед $\frac{\partial^m v(y)}{\partial y_0^m}$ на $S$ обращается в 0. \\
% v, y - это после замены во время распрямления поверхности.
\vspace{14}

$\diamond$ Лекция 6 \\

\begin{definition}
Для уравнения (1) построим многочлен $n+1$ переменной $(\xi_0, \xi_1, \dots, \xi_n)$: $(3) = \sum_{|\alpha| \le m}a_{\alpha}(x)\xi^{\alpha}$. Многочлен (3) называют полным символом уравнения (1), а $(4) = \sum_{|\alpha| = m}a_{\alpha}(x)\xi^{\alpha}$ - главным символом (1).
\end{definition}
$\checkmark$ Иногда в этом определении заменяют $\xi \to i\xi$.
\begin{definition}
Вектор $\xi \in \mathbb{R}^{n+1} \setminus \{\theta\}$ задаёт характеристическое направление уравнения (1) в т. $x$, если главный символ уравнения обращается в 0 на $\xi$. ($\sum_{|\alpha|=m} a_{\alpha}(x)\xi^{\alpha} = 0$)
\end{definition}
% \nabla F || n

% мне ультралень записывать определение характеристического направления для системы.

\subsection{Классификация уравнений и систем УрЧП. Эллиптические, гиперболические и параболические системы.}

\begin{definition}[Эллиптическое уравнение]
Уравнение называется эллиптическим в т. $x$, если $\forall \xi  \in \mathbb{R}^{n+1} \setminus \{\theta\}$ выполнено $\sum_{|\alpha|=m} a_\alpha(x) \xi^\alpha \neq 0$.
\end{definition}
$\checkmark$ Альтернативная формулировка: уравнение не имеет характеристических направлений. \\

\begin{definition}[Гиперболическое уравнение]
Уравнение называется гиперболическим в т. $x$, если его хар. уравнение $\sum_{|\alpha|=m} a_\alpha(x) \xi^\alpha = 0$ имеет только действ. корни отн-но $\xi_0$ при $\xi^{\prime}=\left(\xi_1, \ldots, \xi_n\right) \in \mathbb{R}^{n} \setminus \{\theta\}$ 
\end{definition}

$\diamond$ Лекция 7

\begin{definition}[Параболическое уравнение]
Уравнение называется параболическим (по Петровскому) в т. $x$, если $\exists \, p \in \mathbb{N}$, т. ч. относительно $\xi_0$ все действительные части всех корней уравнения
$$
\sum_{\left|\alpha^{\prime}\right|+p \cdot \alpha_0=m} a_\alpha\left(x_0, x^{\prime}\right) \xi_0^{\alpha_0} \cdot\left(i \xi^{\prime}\right)^{\alpha^{\prime}}=0
$$ отрицательны при всех $|\xi^{\prime}| = 1$, где $\xi^{\prime} \in \mathbb{R}^{n} \setminus \{\theta\}$
\end{definition}

\vspace{14}

Система:
$$
\frac{\partial u_i}{ \partial x_0} = \sum_{j=1}^N \sum_{|\alpha| \le n_j } a_{i j}^\alpha(x) \frac{\partial^\alpha u_j}{\partial x_0^{\alpha_0} \ldots \partial x_n^{\alpha_n}}=f_i(x), i=1,2, \ldots, N \\
$$
\begin{definition}[Эллиптическая система]
называется эллиптической, если $\forall \xi \in \mathbb{R}^{n+1} \setminus \{\theta\}$ выполнено: $\operatorname{det}\left(\sum_{|a|=n_j}a_{i j}^\alpha(x) \xi_0^{\alpha_0} \ldots \cdot \xi_n^{\alpha_n}\right)  \neq 0$. \\
\end{definition}

\begin{definition}[Гиперболическая система]
называется гиперболической в т. $x$, если $\forall  \xi=\left(\xi_1, \ldots, \xi_n\right) \in \mathbb{R}^{n} \setminus \{\theta\}$ в определителе $\operatorname{det}\left(\sum_{|a|=n_j}a_{i j}^\alpha(x) \xi_0^{\alpha_0} \cdot \xi_1^{\alpha_1} \ldots \cdot \xi_n^{\alpha_n}\right)$ все корни относительно действительных переменных $\xi_0$ только действительные. \\
\end{definition}

\begin{definition}[Параболическая система]
называется параболической в т. $x$, если $\forall |\xi^{\prime}| = 1$, где $  \xi^{\prime}=\left(\xi_1, \ldots, \xi_n\right) \in \mathbb{R}^{n} \setminus \{\theta\}$ в определителе $\operatorname{det}\left(\sum_{|a|=2 \cdot m} (-1)^{m} a_{i j}^\alpha(x) \xi_1^{\alpha_1} \ldots \cdot \xi_n^{\alpha_n} -\xi_0 \cdot \delta_{i j} \right)$ все  корни относительно переменных $\xi_0$ имеют отрицательную действительную часть.
\end{definition}

$\checkmark$ Определение в терминах оператора $\mathbb{L}(x,\mathbb{D})$ выходит за рамки настоящего теоретического минимума. 

Линейное уравнение 2 порядка:
$$
\sum_{i, j=0}^n a_{i j}(x) \frac{\partial^2 u}{\partial x_i \partial x_j}+\sum_{i=0}^n f_i(x) \frac{\partial u}{\partial x_i}+c(x) \cdot u=f(x), x \in \mathbb{R}^{n+1} 
$$ \\
В точке $x_0$ вещественная симметрическая матрица: $A_0 = A(x_0) = (a_{ij}(x_0))$ имеет $n + 1$ собственное значение $\lambda_i$. $p$ - положит. собств. значений, $q$ - отрицательных и $r$ - нулевых.
$\bullet$ Система называется эллиптической, если $(p = n + 1)$ или $(q = n + 1)$. \\
$\bullet$ Система называется параболической, если $(p = n, q = 0, r=1)$ или $(p = 0, q = n, r=1)$. \\
$\bullet$ Система называется гиперболической, если $(p = n, q = 1, r=0)$ или $(p = 1, q = n, r= 0)$. \\
$\bullet$ Система называется ультрагиперболической, если $(p \geq 2, q \geq 2, p + q = n + 1)$. \\

$\checkmark$ Линейными преобразованиями эти уравнения приводятся к каноническому виду. \\

Канонический вид \textbf{эллиптич.} уравнения:
$$
\pm \sum_{k=0}^n \frac{\partial^2 \tilde{u}^2}{\partial y_k^2}+\sum_{k=0}^n \tilde{b}_k \cdot \frac{\partial \tilde{u}^2}{\partial y_k}+\tilde{c} \cdot \tilde{u}=\tilde{f}(y)
$$
Канонический вид \textbf{параболич.} уравнения:
$$
\pm \sum_{k=1}^n \frac{\partial^2 \tilde{u}^2}{\partial y_k{ }^2}+\sum_{k=0}^n \tilde{b}_k \cdot \frac{\partial \tilde{u}^2}{\partial y_k}+\tilde{c} \cdot \tilde{u}=\tilde{f}(y)
$$

Канонический вид \textbf{гиперболич.} и \textbf{ультрагиперболич.} уравнения:
$$

& \pm\left(\sum_{k=1}^n \frac{\partial^2 \tilde{u}^2}{\partial y_k^2}-\frac{\partial^2 \tilde{u}}{\partial y_0^2}\right)+\sum_{k=0}^n \tilde{b}_k \cdot \frac{\partial \tilde{u}}{\partial y_k}+\tilde{c} \cdot \tilde{u}=\tilde{f}(y) \\
& 1 \leqslant m \leqslant n-2 \quad \sum_{k=0}^m \frac{\partial^2 \tilde{u}}{\partial y_k^2}-\sum_{k=m+1}^n \frac{\partial^2 \tilde{u}}{\partial y_k^2}=\tilde{f}(y)

$$

\end{document}
